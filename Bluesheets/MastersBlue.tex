\documentclass[blue]{FateDarkDawn}
\begin{document}
\name{\bMaster{}}

You have been selected as a Master in the sixth Holy Grail War. You should write your will, and make arrangements to pass on your family's crest now (if your family is old enough to even have one). Many people would kill to be in your shoes right now. Many people will try to kill you \emph{because} of the shoes you are in. The Holy Grail War is just that, a war. Expect to kill. Expect to be killed. Why? Because \iGrail{} is at stake.

Sometime in the not so distant past, you summoned a servant to help you survive this war. You may have used a relic to summon your servant, targeting a particular heroic spirit, or you may have performed the summoning ritual without one, and gambled. Either way, they are your partner. If for some reason you don't know \emph{who} your servant is, you should probably find out. Better to know as soon as possible who you are dealing with. If you don't already know why they want the grail, you should probably find that out too. Better to know as soon as possible whether they are a being you can work with easily, or if you will be at odds.

Your servant was summoned through space and time by the power of \iGrail{\MYname}, but their physical form is now sustained by mana you supply as a mage. If you could find a way to increase your mana output, your servant's strength would increase dramatically. However, if you were to somehow lose the magical circuits in your body instead, you would no longer be able to supply your servant with mana at all. They will fade away rapidly without a source of mana. Technically, the souls of non-mages can be drained as a source of mana, but it is strictly forbidden by the Clock Tower (the most important coalition of mages in the world) and considered an anathema by all respectable mage lines. The Church may also choose to get involved if innocent civilians are  targeted.

You have three \iCommand{\MYname}s (\iCommand{\MYnumber}). These mark you as a Master. They are also the only way you can gain absolute control over your servant. They are thinking beings who can, and probably will, act independently of a master they do not respect. If you cannot establish a good rapport with your servant, they may ignore your commands in battle, or actively try to cause you harm. There is even one recorded instance of a servant killing their own master. \iCommand{\MYname}s will allow you to give your servant a direct order that they cannot disobey. Be careful about the scope of your command, the more specific the order, the more effective. ``Give me the apple you are holding'' is far more effective at binding a servant than ``Bring me \iGrail{\MYname}'' or ``Listen to me.''

If you lose your servant (ie: They are killed, or your pact with them is severed somehow), you are not immediately eliminated from the War. Only death, or the use of the last of your \iCommand{\MYname}s will strip you of your status. Some servants may have abilities that allow them to remain for an extended period of time after their original master is killed. A master lacking a servant but still in possession of one or more \iCommand{\MYname} may be able to persuade such a servant to form a new pact.

Be careful. Homicidal mages are dangerous enough, but when they are paired with supernatural reincarnations of humanity's greatest heroes and villains, they are no less than a force of nature. No master can expect to survive a one-on-one encounter with a hostile servant. They are just too powerful. The various servant classes are also ranked in power, albeit with some variation, depending on the exact identity of the heroic spirit in question. In general, the berserker class is considered the most powerful, and may be difficult or impossible for a single servant to take down. The caster class is usually identified as the physically weakest class, but more than makes up for it in their ability to manipulate mana, and some spirits in this class may rival the Archer class's ranged combat prowess. Temporary alliances are not unheard of, but never forget that your ally one minute will be your enemy the next.

\begin{itemz}[Goals]
	\item Win \iGrail{} by being th Master of the last remaining Servant.
\end{itemz}

\end{document}
