\documentclass[blue]{FateDarkDawn}
\begin{document}
\name{\bHGW{}}

\bigquote{``Welcome to the Holy Grail Wars. You have been chosen as a Master, or summoned as a Servant. For the uninitiated, or those who find their mage's education to be lacking, let me fill in the gaps. Pay attention. Some of this information may improve your chances of survival.''}{-- \cChurch{\MYname{}}}

As implied by the name, this is the sixth contest for \iGrail{}. This isn't \emph{The} Holy Grail, nevertheless, the Church has presided over these trials for the last couple hundred years. \iGrail{} was created by the three first families, Eizenberg, Tohsaka and Matou, in a ritual so secret it is rumored that the details are lost even to these families.

\iGrail{} is an omnipotent wish granting object. Alas, only the one deemed worth by the Grail may make a wish on the Grail. Approximately every 60 years, \iGrail{} selects seven Masters, summons seven Servants, and appears in Fuyuki City, Japan. The winning pair (master and servant) will have the opportunity to have their wishes granted by \iGrail{}.

Seven masters are chosen for this contest. The first families are given priority you might say. The last few masters to round out the set are often from very young mage families (only a few generations), and are thus at a disadvantage. Each Master is granted {\bf three} \iCommand{\MYname}s. These denote your status as a Master. As long as you still have \iCommand{\MYname}s, you retain your status as Master. Even if your servant is destroyed. It is technically possible for a non-mage to serve as Masters, but without the ability to perform magecraft, they cannot summon a servant. Non-mages with servants have made a pact with a Servant whose Master was killed, or whose

Seven servants are summoned for this contest as well. Each master summons one servant. When the last servant is summoned, the contest is considered begun. The servants are all \emph{heroic sperits} --

\end{document}
