\documentclass[blue]{FateDarkDawn}
\begin{document}
\name{\bHGW{}}

\bigquote{``Welcome to the Holy Grail Wars. You have been chosen as a Master, or summoned as a Servant. For the uninitiated, or those who find their mage's education to be lacking, let me fill in the gaps. Pay attention. Some of this information may improve your chances of survival.''}{-- \cChurch{\MYname{}}}

As implied by the name, this is the sixth contest for \iGrail{\MYname{}}. This isn't \emph{The} Holy Grail, nevertheless, the Church has presided over these trials for the last couple hundred years. \iGrail{\MYname{}} was created by the three first families, the Eizenbergs, Tohsakas and Matous, in a ritual so secret it is rumored that the details are lost even to these families.

\iGrail{\MYname{}} is an omnipotent wish granting object. Alas, only the one deemed worthy by the Grail may make a wish on the Grail. Approximately every 60 years, \iGrail{\MYname{}} selects seven Masters, facilitates the summoning of seven Servants, and appears in Fuyuki City, Japan. The winning pair (master and servant) will have the opportunity to have their wishes granted by \iGrail{\MYname{}}. Technically, the winning pair is the last remaining servant and their master.

Seven masters are chosen for this contest. The first families are given priority you might say. There has not yet been a war where these three families did not all participate. The last few masters to round out the set are often from very young mage families (only a few generations), and are thus usually very weak. Each Master is granted {\bf three} \iCommand{\MYname}s. These denote their status as a Master. As long as they still have \iCommand{\MYname}s, they retain their status as Masters. Even if their servant is destroyed. It is technically possible for a non-mage to serve as Masters, but without the ability to perform magecraft, they cannot summon a servant.

Seven servants are summoned for this contest as well. Each master summons one servant. When the last servant is summoned, the contest is considered begun. The servants are all \emph{heroic spirits} -- identities from the past, present or future who have had, or will have, a major impact on human history. Servants answer the call of the grail because they have some unfulfilled wish from life that they wish fulfilled now thorough the only means available to them: \iGrail{\MYname{}}. There are seven servant classes: Saber, Lancer, Archer, Berserker, Rider, Caster and Assassin. One heroic spirit from each class is summoned in each war, and their class is usually (but not always) pretty obvious. Beserker class servants are usually considered to be the strongest, but are difficult to control. Saber, Rider, and Assassin servants are usually low magic, high physical combat spirits who are vulnerable to overdrawing their mana reserves. The Lancer, Archer and Caster classes often have high magical affinity, but cannot survive long in melee combat with the other classes.

The grail itself will manifest in the physical world when the war nears its completion. There are four places in Fuyuki City where the grail can appear, due to the confluence of spiritual leylines in these locations: the \pChurch{}, the \pTEstate{}, the \pConvention{}, and \pTemple{}.

The Catholic Church has been involved in these wars since the third war. It is the job of the Church Arbiter, sent by the Vatican, to oversee the war. It is \emph{very} important that we keep the secret of magecraft from the unenlightened masses. The church can be called upon to assist if innocents are drawn into the conflict and the police get involved. Since the Catholic church does not condone needless violence, the Church also offers protection to masters who have lost their servant and wish to withdraw from the conflict. It is technically possible for an unpaired servant to make a new pact with a master who has lost their servant, as long as the master still has \iCommand{\MYname}s. Therefore, enemy masters remain enemies, even if their servant is neutralized.

\end{document}
