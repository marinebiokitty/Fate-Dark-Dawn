\documentclass[blue]{FateDarkDawn}
\begin{document}
\name{\bServant{}}

You have been selected as a Servant in the sixth Holy Grail War. You have been given the exceedingly rare opportunity to have a driving wish that went unfulfilled in life granted now. It will never come again -- this is your one shot. Expect to kill. Expect to be killed. Why? Because \iGrail{} is at stake.

Sometime in the not so distant past, you were summoned by a mage who currently acts as your Master. If for some reason you don't know \emph{who} your Master is, you should probably find out. Better to know as soon as possible who you are dealing with. If you don't already know why they want the grail, you should probably find that out too. Better to know as soon as possible whether they are a mage you can work with easily, or if you will be at odds. And if it turns out that you {\bf absolutely cannot} make it work, you could always try to find a different Master. Technically, a servant could kill their own master, but this pretty much destroys your chances of getting the grail. Who would make a pact with servant that had killed their own master?

You were summoned through space and time by the power of \iGrail{\MYname}, but your physical form is now sustained by mana supplied to you by your Master. If you could find a way to increase your Master's mana output, your strength would increase dramatically. However, if they were to somehow lose the magical circuits in their body instead (or die), you would no longer be able to draw mana from them. You will fade away rapidly without a source of mana ({\bf Without a source of mana, you will  fade out and cease to exist in 15 minutes real time unless you know otherwise}). Technically, the souls of non-mages can be drained as a source of mana, but it is strictly forbidden by the Clock Tower (the most important coalition of mages in the world) and considered an anathema by all respectable mage lines. The Church may also choose to get involved if innocent civilians are  targeted. As a servant, you may or may not care, but your master will almost certainly have an opinion on this.

Your master also has three \iCommand{\MYname}s (\iCommand{\MYnumber}). These mark their status as a Master. The \iCommand{\MYname} are also the only way they can gain absolute control over you. Servants are thinking beings who can, and probably will, act independently of a master they do not respect. If your Master doesn't bother to establish good rapport with you, you are under no obligation to show them respect or attention, or help them achieve their goals. \iCommand{\MYname} are the one exception to this. Using a \iCommand{\MYname} will allow a master to give their servant a direct order that the servant cannot disobey in principle. However, the effectiveness of the commands are inversely proportional to their specificity and feasibility. (OOC: As a servant class character, you must judge the specificity and feasibility of the order and act on it in good faith based on these parameters.) The more specific the order, the more effective. ``Give me the apple you are holding'' is far more effective at binding a servant than ``Listen to me.'' The more feasible the order, the more effective. ``Go spy on the Church'' is far more effective than ``Bring me \iGrail{\MYname}.'' NOTE: The commands ``Do not kill me'', and ``stop what you are currently doing'' or equivalent phrases have incredibly high specificity and feasibility and {\bf must} be respected. The duration for which you are compelled to follow these particular commands exceeds total alloted game time (ie: if one of these orders are issued to you, that action is forbidden to you for the rest of game unless countermanded by use of another command seal.) Most other commands should be considered binding for 1 day of in game time (15 minutes real time).

If you lose your master (ie: They are killed, or your pact with them is severed somehow), you are not immediately eliminated from the War. Only death in combat, or running out of mana (no mana source for 15 min real time -- unless you know otherwise) will truly be your end.  A master lacking a servant but still in possession of one or more \iCommand{\MYname} may be willing to form a new pact with a master-less servant. After all, they too desire the grail for their own purposes.

Strategy will be key to your victory in this war. Few servants and masters are foolish enough to engage an enemy without some attempt at reconnaissance beforehand. Knowing your enemies strengths and weaknesses will make your attacks more effective. Knowing their identity will go a long way toward that. Therefore you should guard your own identity with great care. The various servant classes are ranked in general power level, although the identity of the individual heroic spirits may change the rankings somewhat. In general, the berserker class is considered the most powerful, and may be difficult or impossible for a single servant to take down. The caster class is usually identified as the physically weakest class, but more than makes up for it in their ability to manipulate mana. The Archer class obviously dominates in ranged combat, but an assassin's skills can quickly neutralize that advantage. The Rider class usually has unparalleled ability to move around the city, but their animal companions are particularly vulnerable to a Lancer's attack. The Saber class is a powerful all around class, with strong melee combat abilities and usually some magical resistance, but a jack of all trades is master of none. Temporary alliances are not unheard of, but never forget that your ally one minute will be your enemy the next.


\begin{itemz}[Goals]
	\item Win \iGrail{} by being the last Servant (must have a master).
\end{itemz}

\end{document}
